\documentclass[DM,lsstdraft,STR,toc]{lsstdoc}
\input meta.tex

\begin{document}

\def\milestoneName{Alert Generation}
\def\milestoneId{LDM-503-3}
\def\product{LSST Alert Production System}

\setDocCompact{true}

\title[\milestoneId{}~Test Report]{\milestoneId{} (\milestoneName{})~Test Report}
\setDocRef{\lsstDocType-\lsstDocNum}
\setDocDate{\vcsdate}
\setDocUpstreamLocation{\url{https://github.com/lsst/lsst-texmf/examples}}
\author{% `git log --pretty=%an | sort --key=2 | uniq` ?
  Eric C.\ Bellm
}

% Most recent last
\setDocChangeRecord{
	\addtohist{1}{2017-12-08}{Initial version.}{Bellm}
}


\setDocCurator{Eric C. Bellm}
\setDocUpstreamLocation{\url{https://github.com/lsst-dm/\lsstDocType-\lsstDocNum}}
\setDocUpstreamVersion{\vcsrevision}


\setDocAbstract{
This is the test report for \milestoneId{} (\milestoneName{}), an LSST DM level 2 milestone pertaining to the \product{}.
}

\maketitle

\section{Introduction}
\label{sect:intro}

\subsection{Objectives}
\label{sect:objectives}

This document describes the results of tests carried out in late (calendar) 2017 on the \product{} in order to assess progress against the LSST DM level 2 milestone \milestoneId{}.
We report on the success or failure of applicable test cases and assess the state of the software and services tested.

\subsection{Scope}
\label{sect:scope}

The overall test plan for the LSST Data Management system is described in \citeds{LDM-503}.
This document specifically refers to the late (calendar) 2017 milestone \milestoneId{}, which tests the \product{}.
The overall \product{} test specification is defined in \citeds{LDM-534}.
The test plan for \milestoneId{} involves the execution of the entire AP-00
(Small Scale Alert Generation Processing) specification, including the following
test cases:

\begin{description}

  \item[AG-00-00]{Installation of the Alert Generation science payload}
  \item[AG-00-05]{Alert Generation Produces Required Data Products}
  \item[AG-00-10]{Scientific Verification of Processed Visit Images}
  \item[AG-00-15]{Scientific Verification of Difference Images}
  \item[AG-00-20]{Scientific Verification of DIASource Catalog}
  \item[AG-00-25]{Scientific Verification of DIAObject Catalog}

\end{description}

\subsection{System Overview}
\label{sect:systemoverview}

The \product{} is that part of the LSST Data Management system which will be responsible for nightly data processing during LSST operations.
The most prominent example of such processing is the generation of alerts from LSST difference images.
However, the \product{} is also responsible for the distribution and filtering of alerts, excecution of precovery and forced photometry measurements, execution of the MOPS payload, and for the Level 1 Quality Control Service (\citeds{LDM-148}).
The \milestoneId{} milestone focuses only on the alert generation part of the system.

Note that we may broadly think of the \product{} as consisting of two independent parts: the Prompt Processing and Batch Production Services, which provides scheduling and workflow services, and the Science Payloads, which contain the algorithmic content.
The \milestoneId{} milestone exercises only the science payloads.

\subsection{Applicable Documents}
\label{sect:appdocs}
\addtocounter{table}{-1}

\begin{tabular}[htb]{l l}
\citeds{LDM-294} & LSST DM Project Management Plan\\
\citeds{LDM-503} & DM Test Plan\\
\citeds{LDM-533} & \product{} Test Specification\\
\end{tabular}

\subsection{References}
\label{sect:references}

\renewcommand{\refname}{}
\bibliography{lsst,refs,books,refs_ads}

\subsection{Document Overview}
\label{sect:docoverview}

Section \ref{sect:configuration} of this document provides details of the \product{} baseline used for this test, including relevant hardware and software configurations.
Section \ref{sect:personnel} lists the individuals involved in performing the tests.
Section \ref{sect:overview} provides an overview of the test results, while Section \ref{sect:detailed} provides more detailed results from each individual test case.

\section{Test Configuration}
\label{sect:configuration}

\subsection{Documents}

This test report refers to the execution of tests AG-00-00 through AG-00-25 in \citeds{LDM-533}.

\begin{note}
At time of writing, LDM-533 has not been baselined, so a detailed version
number is not available.
\end{note}

\subsection{Hardware}
\label{sect:hwconf}

All tests were executed on systems in the LSST Data Facility.

Software installation (AG-00-00) and scientific analysis work (AG-00-05 through DRP-00-25) were carried out on \texttt{lsst-dev01.ncsa.illinois.edu}.
At time of text execution, this was a Dell PowerEdge R730 with 24 physical Intel Xeon E5-2690v3 CPU cores at 2.60\,GHz and 256\,GB of RAM.

Bulk data processing (part of AG-00-05) was carried out on the LSST Verification Cluster (VC).
At the time of test execution, the VC provided 48 Dell C6320 nodes, each with 24 physical Intel Xeon E5-2680v3 CPU cores at 2.50\,GHz and 128\,GB of RAM.

\subsection{Software}
\label{sect:swconf}

All systems used for testing --- including both \texttt{lsst-dev01} and the VC nodes --- were running CentOS Linux release 7.4.1708.
The \texttt{devtoolset-6}\footnote{\url{https://access.redhat.com/documentation/en-us/red_hat_developer_toolset/6/html/6.0_release_notes/}} toolchain, including GCC\footnote{\url{https://gcc.gnu.org/}} version 6.3.1, was enabled throughout these tests.

\subsection{Input Data}
\label{sect:inputdata}

Input data for all tests was based on the DeCAM “HiTS” dataset, as described in \citeds{LDM-533}.

\section{Personnel}
\label{sect:personnel}

All test cases were executed by Eric Bellm (University of Washington).

\newpage

\section{Overview of the Test Results}
\label{sect:overview}

\subsection{Summary Table}
\label{sect:summarytable}

\begin{longtable} {|p{0.2\textwidth}|p{0.2\textwidth}|p{0.6\textwidth}|}\hline
{\bf TEST CASE ID} & {\bf PASS/FAIL} & {\bf COMMENTS} \\\hline
AG-00-00 & & comment \\\hline
AG-00-05 & & comment \\\hline
AG-00-10 & & comment \\\hline
AG-00-15 & & comment \\\hline
AG-00-20 & & comment \\\hline
AG-00-25 & & comment \\\hline
\end{longtable}

\subsection{Overall Assessment \label{sect:overallassessment}}
\begin{itemize}
\item Provide an overall assessment of the software as demonstrated by the test results in this report
\item Identify any remaining deficiencies, limitations or constraints that were detected by the testing performed
\item For each remaining deficiency, limitation or constraint describe:
\begin{itemize}
\item Its impact on software and system performance, including identification of requirements not met
\item The impact on software and system design to correct it
\item A recommended solution/approach for correcting it
\item Mantis issue
\end{itemize}
\end{itemize}

\subsection{Impact of Test Environment}
\label{sect:impact}

\subsection{Recommended Improvements}
\label{sect:recommendations}


\section{Detailed Test Results}
\label{sect:detailed}

\subsection{AG-00-00}

The string \texttt{Ok} was returned when executing

\begin{verbatim}
  $ python bin/compare expected/Linux64/detected-sources.txt
\end{verbatim}

after running the \texttt{demo.sh} on both the cluster head node (\texttt{lsst-dev01}) and on an example compute node.

One \texttt{pytest} error was encountered by \texttt{ap\_verify}:

\begin{verbatim}
______________________ MeasureRuntimeTestSuite.test_valid ______________________

self = <test_profiling.MeasureRuntimeTestSuite testMethod=test_valid>

    def test_valid(self):
        """Verify that timing information can be recovered.
            """
        meas = measure_runtime(self.task.getFullMetadata(), task_name='isr', metric_name='ip_isr.IsrTime')
        self.assertIsInstance(meas, Measurement)
        self.assertEqual(meas.metric_name, lsst.verify.Name(metric='ip_isr.IsrTime'))
>       self.assertGreater(meas.quantity, 0.0 * u.second)
E       AssertionError: <Quantity 0.0 s> not greater than <Quantity 0.0 s>

tests/test_profiling.py:54: AssertionError
============================= 29 tests deselected ==============================\end{verbatim}

This is a unit test limitation already reported 
in \texttt{DM-12848} and does not affect the functionality of the package.

\subsection{AG-00-05}
\subsection{AG-00-10}
\subsection{AG-00-15}
\subsection{AG-00-20}
\subsection{AG-00-25}

\appendix

\newpage
\section{Detailed tests results}

\end{document}
