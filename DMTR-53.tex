\documentclass[DM,lsstdraft,STR,toc]{lsstdoc}
\input meta.tex

\begin{document}

\def\milestoneName{Alert Generation}
\def\milestoneId{LDM-503-3}
\def\product{LSST Alert Production System}

\setDocCompact{true}

\title[\milestoneId{}~Test Report]{\milestoneId{} (\milestoneName{})~Test Report}
\setDocRef{\lsstDocType-\lsstDocNum}
\setDocDate{\vcsdate}
\setDocUpstreamLocation{\url{https://github.com/lsst/lsst-texmf/examples}}
\author{% `git log --pretty=%an | sort --key=2 | uniq` ?
  Eric C.\ Bellm
}

% Most recent last
\setDocChangeRecord{
	\addtohist{1}{2017-12-08}{Initial version.}{Bellm}
}


\setDocCurator{Eric C. Bellm}
\setDocUpstreamLocation{\url{https://github.com/lsst-dm/\lsstDocType-\lsstDocNum}}
\setDocUpstreamVersion{\vcsrevision}


\setDocAbstract{
This is the test report for \milestoneId{} (\milestoneName{}), an LSST DM level 2 milestone pertaining to the \product{}.
}

\maketitle

\section{Introduction}
\label{sect:intro}

\subsection{Objectives}
\label{sect:objectives}

This document describes the results of tests carried out in late (calendar) 2017 on the \product{} in order to assess progress against the LSST DM level 2 milestone \milestoneId{}.
We report on the success or failure of applicable test cases and assess the state of the software and services tested.

\subsection{Scope}
\label{sect:scope}

The overall test plan for the LSST Data Management system is described in \citeds{LDM-503}.
This document specifically refers to the late (calendar) 2017 milestone \milestoneId{}, which tests the \product{}.
The overall \product{} test specification is defined in \citeds{LDM-534}.
The test plan for \milestoneId{} involves the execution of the entire AP-00
(Small Scale Alert Generation Processing) specification, including the following
test cases:

\begin{description}

  \item[AG-00-00]{Installation of the Alert Generation science payload}
  \item[AG-00-05]{Alert Generation Produces Required Data Products}
  \item[AG-00-10]{Scientific Verification of Processed Visit Images}
  \item[AG-00-15]{Scientific Verification of Difference Images}
  \item[AG-00-20]{Scientific Verification of DIASource Catalog}
  \item[AG-00-25]{Scientific Verification of DIAObject Catalog}

\end{description}

\subsection{System Overview}
\label{sect:systemoverview}

The \product{} is that part of the LSST Data Management system which will be responsible for nightly data processing during LSST operations.
The most prominent example of such processing is the generation of alerts from LSST difference images.
However, the \product{} is also responsible for the distribution and filtering of alerts, excecution of precovery and forced photometry measurements, execution of the MOPS payload, and for the Level 1 Quality Control Service (\citeds{LDM-148}).
The \milestoneId{} milestone focuses only on the alert generation part of the system.

Note that we may broadly think of the \product{} as consisting of two independent parts: the Prompt Processing and Batch Production Services, which provides scheduling and workflow services, and the Science Payloads, which contain the algorithmic content.
The \milestoneId{} milestone exercises only the science payloads.

\subsection{Applicable Documents}
\label{sect:appdocs}
\addtocounter{table}{-1}

\begin{tabular}[htb]{l l}
\citeds{LDM-294} & LSST DM Project Management Plan\\
\citeds{LDM-503} & DM Test Plan\\
\citeds{LDM-533} & \product{} Test Specification\\
\end{tabular}

\subsection{References}
\label{sect:references}

\renewcommand{\refname}{}
\bibliography{lsst,refs,books,refs_ads}

\subsection{Document Overview}
\label{sect:docoverview}

Section \ref{sect:configuration} of this document provides details of the \product{} baseline used for this test, including relevant hardware and software configurations.
Section \ref{sect:personnel} lists the individuals involved in performing the tests.
Section \ref{sect:overview} provides an overview of the test results, while Section \ref{sect:detailed} provides more detailed results from each individual test case.

\section{Test Configuration}
\label{sect:configuration}

\subsection{Documents}

This test report refers to the execution of tests AG-00-00 through AG-00-25 in \citeds{LDM-533}.

\begin{note}
At time of writing, LDM-533 has not been baselined, so a detailed version
number is not available.
\end{note}

\subsection{Hardware}
\label{sect:hwconf}

All tests were executed on systems in the LSST Data Facility.

Software installation (AG-00-00) and scientific analysis work (AG-00-05 through DRP-00-25) were carried out on \texttt{lsst-dev01.ncsa.illinois.edu}.
At time of text execution, this was a Dell PowerEdge R730 with 24 physical Intel Xeon E5-2690v3 CPU cores at 2.60\,GHz and 256\,GB of RAM.

Bulk data processing (part of AG-00-05) was carried out on the LSST Verification Cluster (VC).
At the time of test execution, the VC provided 48 Dell C6320 nodes, each with 24 physical Intel Xeon E5-2680v3 CPU cores at 2.50\,GHz and 128\,GB of RAM.

\subsection{Software}
\label{sect:swconf}

All systems used for testing --- including both \texttt{lsst-dev01} and the VC nodes --- were running CentOS Linux release 7.4.1708.
The \texttt{devtoolset-6}\footnote{\url{https://access.redhat.com/documentation/en-us/red_hat_developer_toolset/6/html/6.0_release_notes/}} toolchain, including GCC\footnote{\url{https://gcc.gnu.org/}} version 6.3.1, was enabled throughout these tests.

\subsection{Input Data}
\label{sect:inputdata}

Input data for all tests was based on the DeCAM “HiTS” dataset, as described in \citeds{LDM-533}.

\section{Personnel}
\label{sect:personnel}

All test cases were executed by Eric Bellm (University of Washington).

\newpage

\section{Overview of the Test Results}
\label{sect:overview}

\subsection{Summary Table}
\label{sect:summarytable}

\begin{longtable} {|p{0.2\textwidth}|p{0.2\textwidth}|p{0.6\textwidth}|}\hline
{\bf TEST CASE ID} & {\bf PASS/FAIL} & {\bf COMMENTS} \\\hline
AG-00-00 & PASSED & \\\hline
AG-00-05 & PARTIALLY PASSED & Incomplete execution due to misconfigured SLURM file; one processing failure. \\\hline
AG-00-10 & PARTIALLY PASSED & Some improvements in instrument signature remove needed. \\\hline
AG-00-15 & FAILED & Missing provenance information for difference images. Some mask propagation issues. \\\hline
AG-00-20 & FAILED & Many required DIASource catalog columns are not implemented. \\\hline
AG-00-25 & FAILED & Many required DIAObject catalog columns are not implemented. \\\hline
\end{longtable}

\subsection{Overall Assessment \label{sect:overallassessment}}
\begin{itemize}
	\item Installation of the Alert Production pipelines was accomplished without difficulty.
\item Execution of the Alert Production pipelines was generally successful.  
A minor subset of the expected products were not generated due to misconfiguration of the batch processing system, but this is not the mode of execution planned for operations.
\item The resulting Processed Visit Images are appropriately processed and maskeed, in general, but some improvements in ISR are necessary.  In particular, crosstalk corrections are not applied, and some masks are missing.
\item Required provenance information for difference images are absent, and should be corrected once SuperTask and the new Butler are finalized.
\item The DIASource catalogs only contain a subset of the many attributes required (see Table \ref{tab:diasrc}).
\item The DIAObject catalogs at present only implement raw source association; all ``lightcurve'' features and related summary/aggregate information is missing.
%\item For each remaining deficiency, limitation or constraint describe:
%\begin{itemize}
%\item Its impact on software and system performance, including identification of requirements not met
%\item The impact on software and system design to correct it
%\item A recommended solution/approach for correcting it
%\end{itemize}
\end{itemize}

%\subsection{Impact of Test Environment}
%\label{sect:impact}

\subsection{Recommended Improvements}
\label{sect:recommendations}

\begin{itemize}

	\item{The Alert Production pipelines should be integrated into \texttt{lsst\_distrib} and high-level documentation provided through \url{pipelines.lsst.io}.}

	\item{Crosstalk corrections should be integrated into the \texttt{ap\_pipe} pipeline.}
	\item{Provenance information for difference images should be provided.}
	\item{DIASource and DIAObject catalogs should be enhanced to contain all required attributes.}
	\item{DMS-REQ-0270 is untestable as written.  
		Further clarification of the criteria by which sub-threshold 
		DIASources are selected to be stored should be defined.}
	\item{DMS-REQ-0331 should be clarified or enhanced---which models should have goodness of fit stored?  Which color-color diagrams?  Are there other quantities of interested that should be required?}

\end{itemize}

\section{Detailed Test Results}
\label{sect:detailed}

\subsection{AG-00-00}

\textit{The Alert Generation Science Payload is available from documented channels: PASSED.}

We note that general-user documentation was in review at the time of this test (\jira{DM-11592}).

\textit{The Alert Generation science payload can be installed on LSST Data Facility-managed systems: PASSED.}

The string \texttt{Ok} was returned when executing

\begin{verbatim}
  $ python bin/compare expected/Linux64/detected-sources.txt
\end{verbatim}

after running the \texttt{demo.sh} on both the cluster head node (\texttt{lsst-dev01}) and on an example compute node.

One \texttt{pytest} error was encountered by \texttt{ap\_verify}:

\begin{verbatim}
______________________ MeasureRuntimeTestSuite.test_valid ______________________

self = <test_profiling.MeasureRuntimeTestSuite testMethod=test_valid>

    def test_valid(self):
        """Verify that timing information can be recovered.
            """
        meas = measure_runtime(self.task.getFullMetadata(), task_name='isr', metric_name='ip_isr.IsrTime')
        self.assertIsInstance(meas, Measurement)
        self.assertEqual(meas.metric_name, lsst.verify.Name(metric='ip_isr.IsrTime'))
>       self.assertGreater(meas.quantity, 0.0 * u.second)
E       AssertionError: <Quantity 0.0 s> not greater than <Quantity 0.0 s>

tests/test_profiling.py:54: AssertionError
============================= 29 tests deselected ==============================\end{verbatim}

This is a unit test limitation already reported 
in \jira{DM-12848} and does not affect the functionality of the package.

\subsection{AG-00-05}

\textit{Execution of the Alert Generation Pipelines produces the expected data products: PARTIALLY PASSED}

We performed a Butler \texttt{get} for \texttt{calexp}, \texttt{deepDiff\_differenceExp}, and \texttt{deepDiff\_diaSrc} dataset types using 
dataIds constructed all visits and ccds listed in the appendix of \citeds{LDM-533}.

We found several dataIds where products were not generated:  

\begin{itemize}
	\item For visits 411321, 411371, and 411422, ccds 48--60 and 62 were missing all three of \texttt{calexp}, \texttt{deepDiff\_differenceExp}, and \texttt{deepDiff\_diaSrc} products.  This failure was due to the time limit being reached on the slurm execution script (which looped sequentially over ccds for each visit).
	\item For visit 411657, ccd 47 only was missing all three products.  Inspecting the log revealed an error:
		\begin{verbatim}  
File "/software/lsstsw/stack3_20171023/stack/miniconda3-4.3.21-10a4fa6/
Linux64/meas_algorithms/14.0-7-g23fdbe95+15/python/lsst/meas/algorithms/
measureApCorr.py", line 251, in run
    (name, len(subset2), self.config.minDegreesOfFreedom+1))
RuntimeError: Unable to measure aperture correction for required algorithm 
'base_PsfFlux': only 0 sources, but require at least 2.
\end{verbatim}
\end{itemize}

We note the need for increased SLURM timelimits.

The exception raised for visit 411657, ccd 47 was due to a failure to select
sufficient stars for PSF modelling; we have created \jira{DM-13136} to 
investigate the root cause of this failure.

We inspected the \texttt{DIAObject} table by opening the 
\texttt{l1db/association.db} sqlite database produced by \texttt{ap\_pipe} 
with \texttt{sqlite3}.  Issuing the query \texttt{select * from dia\_objects limit 10;} confirmed that the database was non-empty.


\subsection{AG-00-10}

\textit{Generated PVIs include science, mask, and variance images; background model; zeropoint; PSF, and WCS: PASSED}

For all of the generated PVIs, we used Butler calls to check for the existence
image, mask, and variance planes; zeropoints, PSFs, and WCS. The background models are persisted separately, and were checked with appropriate Butler calls as well.

All dataIds where data was generated in AG-00-05 had all of the required components.

\textit{Masks and background are appropriately applied: PARTIALLY PASSED}

We chose two visits and five ccds at random:
visit 411673, ccds 15, 32, 45, 55, and 57, and
visit 410915, ccds 3,  8, 23, 41, and 44.

We used the Butler and DS9 to examine these PVIs.  We noted a few unmasked 
bad columns as well as the effect of crosstalk; crosstalk correction for DECam
was not enabled at the time of this processing because the implementation 
ticket was not yet completed (\jira{DM-10299}).  We created a new ticket
to examine the bad column masking (\jira{DM-13081}).

\subsection{AG-00-15}

\textit{Difference images include provenance information relating to the 
identity of the input exposures and the PSF matching kernel:  FAILED}

No direct provenance information about the identity of the calexp and template
exposures is available through current Butler interfaces.  While the input
calexp can be reconstructed by matching dataIds from the processed image
repository, determining the template applied requires a new call to
\texttt{getCoaddAsTemplateTask}, with concomitant restrictions on 
stack versioning, template repository association, etc.
We have created a new ticket (\jira{DM-13085}) to improve this handling.

For each difference image, we used Butler calls to confirm that the PSF matching kernel was stored.  No failures were reported from the generated images.

\textit{Masks are correctly propagated from the input image: PARTIALLY PASSED}

We used the visits and ccds from AG-00-10:
visit 411673, ccds 15, 32, 45, 55, and 57, and
visit 410915, ccds 3,  8, 23, 41, and 44.

We used the Butler and DS9 to compare the PVIs to the resulting differences. 
The difference image masks were often modified relative to the PVIs, making 
direct comparison more difficult.  In general the masks seemed to be propagated,
although not all masks were---cosmic ray masks, for instance, did not appear 
in the difference image.  This behavior was identified in concurrent 
DRP milestone testing (DM-503-2) and has since been fixed (\jira{DM-9953}).


\subsection{AG-00-20}

\textit{Measurements are Presented in Flux Units: PASSED }

For all of the generated DIASource catalogs, we used Butler calls to check for
the existence of \texttt{PsfFlux} or \texttt{ApFlux} entries.  No failures were
reported from the generated catalogs.


\textit{Each DIASource record contains required attributes: FAILED}

DMS-REQ-0269 requires many attributes, which are summarized in Table
\ref{tab:diasrc}.  We determined which attributes were present by manually
examining a randomly selected DIASource record retrieved from the Butler.

\begin{table}[h]
	\begin{tabular}{|p{0.4\textwidth}|c|p{0.4\textwidth}|}
		\hline
		Attribute & Pass/Fail & Comment \\
		\hline\hline
the identity of the Difference Exposure from which it was derived & FAIL & Only implicitly available by reconstructing dataId \\
		\hline
	the identity of the associated SSObject, if any & FAIL & Out of scope for this test; but no column is present \\
		\hline
		the identity of the parent Source from which this DIASource has been deblended, if any & PASS & \texttt{parent} column is present \\
		\hline
		epoch of the observation & FAIL & not present in the catalog (requires denormalization from higher-level metadata) \\
		\hline
	focal plane position centroid and error (pixel) & PASS &
		\texttt{base\_SdssCentroid\_x},
		\texttt{base\_SdssCentroid\_y},
		\texttt{base\_SdssCentroid\_xSigma}, \&
		\texttt{base\_SdssCentroid\_ySigma} \\
		\hline
		sky position and associated error (radec) & FAIL & \texttt{coord\_ra} and \texttt{coord\_dec} are present, but no error is reported \\
		\hline
	SNR of the detection & FAIL & no SNR is reported \\
		\hline
		calibrated PS flux and associated error & PASS & \texttt{base\_PsfFlux\_flux} and \texttt{base\_PsfFlux\_fluxSigma} \\
		\hline
	likelihood of the observed data given the PS model & False & not present in the catalog \\
		\hline
		calibrated aperture flux and associated error & PASS & several values of \texttt{base\_CircularApertureFlux\_*} are present with errors \\
		\hline
	calibrated flux and associated error for a trailed source model & FAIL & not present in the catalog \\
		\hline
	length and angle of the trail & FAIL & not present in the catalog \\
		\hline
		flux and associated parameters for a dipole model & PASS & several dipole flux values are present \\
		\hline
		parameters of an adaptive shape measurement and associated error & PASS & \texttt{base\_SdssShape\_*} \\
		\hline
	a measure of source extendedness & FAIL & not present in the catalog \\
		\hline
	the estimated background at the position of the object in the template image with associated uncertainty & FAIL & not present in the catalog \\
		\hline
	a measure of spuriousness & FAIL &  not present in the catalog \\
		\hline
	flags indicating problems encountered while computing the aforementioned attributes & PASS & many flags are present \\
		\hline
  calibrated flux and associated error for the DIAsource as measured on the Visit image & FAIL & not present in the catalog \\
		\hline
\end{tabular}
	\caption{Required DIASource attributes for AG-00-20 \label{tab:diasrc}}
\end{table}


\textit{Faint DIASources satisfiying additional criteria are stored: DEVIATION}

No specification for the ``additional criteria'' necessary to store 
sub-threshold DIASources are available at this time; 
this is a failure of requirements rather than of the pipelines.

A new ticket, \jira{DM-12967}, has been created to capture this issue.

\textit{Relevant derived quantities are provided in pre-computed columns: PASSED}

Examination of the DIASource record shows two entries,
\texttt{ip\_diffim\_PsfDipoleFlux\_chi2dof} and
\texttt{ip\_diffim\_DipoleFit\_chi2dof}, which provide goodness of fit measures
as required.  Color-color diagrams are not relevant for single-visit DIASource
catalogs. The requirement does not specify an exhaustive list of relevant
derived quantities and hence cannot be fully tested as written.

\subsection{AG-00-25}

\textit{Measurements are Presented in Flux Units: DEVIATION}

No measurements (or aggregated measurements, such as means) 
were contained in the the DIAObject table as implemented at the 
time of the test, so we could not test this case.

\textit{Each DIAObject record contains required attributes: FAILED}

From DMS-REQ-0271, the required attributes include: a unique identifier, 
identifiers of the nearest stars and galaxies in the Object catalog, 
a probability that the DIAObject is the Object, and a set of DIAObject 
properties.

We verified that all DIAObjects had a unique identifier by comparing the 
output of \texttt{select count(distinct(id)) from dia\_objects;} and
\texttt{select count(id) from dia\_objects;} on the Level 1 database.  
Both returned 546688 objects.

The attributes linking to the Object catalog are out of scope for this test, as
they require joint processing with DRP.

By examining the schema of the implemented Level 1 database, we determined that only one DIAObject property is implemented at this time, a count of the number
of DIASources matched to it.  However, all values in the column were 1, which 
is not expected for a real astrophysical dataset.  This was identified as due
to a bug in the association code and fixed in \jira{DM-13052}; the DIAObject 
association was performed correctly but the property was not updated.


From DMS-REQ-0272, the required summary attributes include: 
position, proper motion and parallax, magnitudes and errors per passband; 
mean forced photometry flux and errors; variability measures; and flags.

None of these attributes are contained in the Level 1 database 
implemented as of this test.


\textit{Each DIAObject record characterizes varability using DIASources contained in the required time interval: DEVIATION}

The timeseries features to be computed on DIAObjects have not been finalized
(cf.\ \jira{DM-11962}), and a minimal subset has yet to be implemented.  Accordingly
as of this writing there are no variability metrics computed on the DIAObjects 
and we cannot test the time interval.

\textit{Relevant derived quantities are provided in pre-computed columns: FAILED}

No derived quantities of any type are contained in the Level 1 database 
implemented as of this test, although we note that which
columns are required is poorly specified.

\section{Deviations from test cases/procedures}

\subsection{AG-00-00}

None.



\end{document}
